\frame { \frametitle{Hardware and Software}
A computing system is built from hardware and software

\begin{block}{The hardware}
is the physical medium and contains CPU, memory, keyboard, display, disks, 
ethernet interfaces etc.
\end{block}

\begin{block}{The software}
is a set of computer programs and contains operating system, compilers, editors,
Fortran programs, etc.
\end{block}
}

\frame { \frametitle{Fortran Compiler}
\begin{block}{Compiler - from Wikipedia}
A compiler is a computer program (or a set of programs) that transforms source 
code written in a programming language (the source language) into another computer 
language (the target language), with the latter often having a binary form known 
as object code. The most common reason for converting source code is to create an 
executable program. 
\end{block}

The FORTRAN team led by John Backus at IBM introduced the first unambiguously 
complete compiler in 1957. 

\begin{block}{Some popular Fortran compilers}
GNU Fortran(gfortran), Intel Fortran(ifort), 
G95(g95), IBM(xlf90), Cray(ftn), Portland Group Fortran (pgf90)
\end{block}
}

\frame { \frametitle{Fortran Source Code}
\begin{itemize}
 \item Fortran 90 and later versions support free format source code.
 \item Fortran source code is in ASCII text and can be written in any text editor.
 \item Fortran source code is case \textbf{insensitive}. PROGRAM is the
same as Program and pRoGrAm.
 \item Use whatever convention you are comfortable with and be consistent throughout.
 \item Comments in Fortran 90 source code start with an exclamation mark (!) except in a character string.
 Comments help to enhance the readability of your code.
\end{itemize}
}

\frame { \frametitle{Variables}
Variables are the fundamental building blocks of any program
\begin{itemize}
 \item A variable name may consist of up to 31 alphanumeric characters and 
 underscores, of which the first character must be a letter.
 \item There are no reserved words in Fortran.
 \item Variable names must begin with a letter and should not contain a space.
\end{itemize}
}

\frame { \frametitle{Variable Types}
\begin{block}{Intrinsic data types}
\begin{itemize}
 \item INTEGER: exact whole numbers
 \item REAL: real, franctional numbers
 \item COMPLEX: complex, fractional numbers
 \item LOGICAL: boolean values
 \item CHARACTER: strings
\end{itemize}
\end{block}
\begin{itemize}
 \item Users can define additional types.
 \item REAL is a single-precision floating-point number.
 \item FORTRAN provides DOUBLE PRECISION data type for double precision REAL. 
       This is obsolete but is still found in many programs.
\end{itemize}
}

\begin{frame}[fragile]{Explicit and Implicit Typing}
\begin{block}{Implicit typing of variables}
$$\underbrace{ABCDEFGH}_{real}\overbrace{IJKLMN}^{integer}\underbrace{OPQRSTUVWXYZ}_{real}$$
\end{block}
\begin{lstlisting}
IMPLICIT DOUBLE PRECISION (a-h,o-z)
\end{lstlisting}
\begin{itemize}
 \item it is highly recommended to explicitly declare all variable and
avoid implict typing using the statement.
\begin{lstlisting}
IMPLICIT NONE
\end{lstlisting}
 \item the IMPLICIT statement must precede all variable declarations.
\end{itemize}
\end{frame}

\begin{frame}[fragile]{Contants - I}
\begin{block}{Integer}
\begin{lstlisting}
242, -2341, 290223
\end{lstlisting}
\end{block}
\begin{block}{Real (single precision)}
\begin{lstlisting}
1.03, 3.51e23, -8.201
\end{lstlisting}
\end{block}
\begin{block}{Real (double precision)}
\begin{lstlisting}
1.03d0, 3.51d23, -8.201d0
\end{lstlisting}
\end{block}
\end{frame}

\begin{frame}[fragile]{Constants - II}
\begin{block}{Complex (single precision)}
\begin{lstlisting}
(1.0,0.0), (-2.5e-5, 3.0e-6)
\end{lstlisting}
\end{block}
\begin{block}{Complex (double precision)}
\begin{lstlisting}
(1.0d0,0.0d0), (-2.5d-5, 3.0d-6)
\end{lstlisting}
\end{block}
\begin{block}{Logical}
\begin{lstlisting}
.True., .False.
\end{lstlisting}
\end{block}
\begin{block}{Character}
\begin{lstlisting}
"Hello World!", "Is pi 3.1415926?"
\end{lstlisting}
\end{block}
\end{frame}

\begin{frame}[fragile]{Variable Declarations - I}
\begin{block}{Numerical variables}
\begin{lstlisting}
INTEGER :: i, j = 2
REAL :: a, b = 4.d0
COMPLEX :: x, y
\end{lstlisting}
\end{block}
\begin{block}{Constant variables}
\begin{lstlisting}
INTEGER, PARAMETER :: j = 2
REAL, PARAMETER :: pi = 3.14159265
COMPLEX, PARAMETER :: ci = (0.d0,1.d0)
\end{lstlisting}
\end{block}
\end{frame}

\begin{frame}[fragile]{Variable Declarations - II}
\begin{block}{Logical variables}
\begin{lstlisting}
LOGICAL :: l, flag=.true.
\end{lstlisting}
\end{block}
\begin{block}{Character variables}
The length of a character variable is set with \textbf{LEN}, which is 
the maximum number of characters (including space) the variable will store.
By default, \textbf{LEN=1} thus only the first character is saved in memory 
if \textbf{LEN} is not specified.
\begin{lstlisting}
CHARACTER(LEN=10) :: a
CHARACTER :: ans = 'yes' !stored as ans='y'
\end{lstlisting}
\end{block}
\end{frame}

\begin{frame}[fragile]{Array Variable}
Arrays (or matrices) hold a collection of different values at the
same time. Individual elements are accessed by subscripting the array.
Fortran arrays are defined with the keyword \textbf{DIMENSION(lower bound: upper bound)} 
\begin{block}{Arrays}
\begin{lstlisting}
INTEGER, DIMENSION(1:106) :: atomic_number
REAL, DIMENSION(3, 0:5, -10:10) :: values
CHARACTER(LEN=3), DIMENSION(12) :: months
\end{lstlisting}
\end{block}
In Fortran, arrays can have up to seven dimensions. Fortran arrays are column major.
\end{frame}

\frame { \frametitle{KIND Parameter for Variables}
\begin{itemize}
 \item Fortran 90 introduced \textbf{KIND} parameters to parameterize the
selection of different possible machine representations for each
intrinsic data types.
 \item The \textbf{KIND} parameter is an integer which is processor dependent.
 \item There are only 2(or 3) kinds of reals: 4-byte, 8-byte (and 16-byte),
respectively known as single, double (and quadruple) precision.
 \item The corresponding \textbf{KIND} numbers are 4, 8 and 16 for most compilers.
\end{itemize}
}

\begin{frame}[fragile]{Operators}
\begin{block}{Arithmetic Operators}
\begin{lstlisting}
+,-,*,/,**
\end{lstlisting}
\end{block}
\begin{block}{Relational Operators}
\begin{lstlisting}
==, <. <=, >, >=, /=
\end{lstlisting}
\end{block}
\begin{block}{Logical Operators}
\begin{lstlisting}
.AND., .OR., .NOT., .EQV., .NEQV.
\end{lstlisting}
\end{block}
\begin{block}{Character Concatenation Operator}
\begin{lstlisting}
//
\end{lstlisting}
\end{block}
\end{frame}


\begin{frame}[fragile]{Operators Evaluations}
\begin{itemize}
 \item All operator evaluations on variables is carried out from left-to-right.
 \item Arithmetic operators have a highest precedence while logical operators have the lowest precedence
 \item The order of operator precedence can be changed using parenthesis, ’(’ and ’)’
 \item A user can define his/her own operators.
 \item Extra parenthesis could be added to enhance readability and avoid mistakes.
\end{itemize}
\end{frame}

\begin{frame}[fragile]{Expressions}
An expression is a combination of one or more operands, zero or
more operators, and zero or more pairs of parentheses.
\begin{block}{Arithmetic expressions}
\begin{lstlisting}
y + 1.0 - x, sin(x) + y
\end{lstlisting}
\end{block}
\begin{block}{Relational expressions}
\begin{lstlisting}
a .and. b, c .neqv. d
\end{lstlisting}
\end{block}
\begin{block}{Character expressions}
\begin{lstlisting}
'hello' // 'world', 'ab' // 'xy'
\end{lstlisting}
\end{block}
\end{frame}

\begin{frame}[fragile]{Statements}
A statement is a complete instruction. Statements may be classified into two types: 
executable and non-executable.
\begin{itemize}
 \item Executable statements are those which are executed at runtime.
 \item Non-executale statements provide information to compilers.
 \item If a statement is too long, it may be continued by the ending the
line with an \textbf{ampersand (\&)}.
 \item Max number of characters (including spaces) in a line is 132
though it’s standard practice to have a line with up to 80
 \item Multiple statements can be written on the same line provided the
statements are separated by a semicolon.
\end{itemize}
\end{frame}

\begin{frame}[fragile]{Intrinsic Functions - I}
Fortran provides many commonly used functions, called intrinsic functions.
\begin{block}{Numerical functions}
\begin{lstlisting}
ABS(A), CEILING(A), FLOOR(A), MAX(A,B), 
MIN(A,B), MOD(I,J), SQRT(A), EXP(A), LOG(A), 
LOG10(A), INT(A), REAL(A), DBLE(A), 
CMPLX(A[,B]), AIMAG(A)
\end{lstlisting}
\end{block}
\begin{block}{Math functions}
\begin{lstlisting}
SIN(A), COS(A), TAN(A), ASIN(A), 
ACOS(A), ATAN(A), ATAN2(A,B), SINH(A), 
COSH(A), TANH(A)
\end{lstlisting}
\end{block}
\end{frame}

\begin{frame}[fragile]{Intrinsic Functions - II}
\begin{block}{Character functions}
\begin{lstlisting}
LEN(S), LEN_TRIM(S), LGE(S1,S2), LGT(S1,S2),
LLE(S1,S2), LLT(S1,S2), ADJUSTL(S), 
ADJUSTR(S), REPEAT(S, N), SCAN(S, C), TRIM(S)
\end{lstlisting}
\end{block}
\begin{block}{Array functions}
\begin{lstlisting}
SIZE(A[,N]), SUM(A[,N]), PRODUCT(A[,N]), 
TRSNSPOSE(A), DOT_PRODUCT(A,B), MATMUL(A,B),
CONJG(X)
\end{lstlisting}
\end{block}
\end{frame}
