\begin{frame}[fragile]{Hello World in Fortran}
\begin{block}{Source code helloworld.f90}
\begin{lstlisting}
program hello
  print *, 'Hello World!'
end program hello
\end{lstlisting}
\end{block}

\begin{block}{Compile and run}
\begin{lstlisting}
$gfortran -o helloworld helloworld.f90
$./helloworld
Hello World!
\end{lstlisting}
\end{block}
\end{frame}

\frame { \frametitle{What is Fortran?}
\begin{itemize}
 \item Fortran (formerly FORTRAN, derived from \textbf{FOR}mula \textbf{TRAN}slation) is a 
general-purpose, imperative programming language that is especially suited to 
numeric computation and scientific computing. 
 \item Originally developed by IBM in the 1950s for scientific and engineering applications.
 \item Widely used in computationally intensive areas such as numerical weather prediction, 
 finite element analysis, etc.
 \item It has been a popular language for high-performance computing and is used for programs 
that benchmark and rank the world's fastest supercomputers.
\end{itemize}
}

\frame { \frametitle{History of Fortran}
\begin{itemize}
\item FORTRAN — first released by IBM in 1957
\item FORTRAN II — released by IBM in 1958
\item FORTRAN IV — released in 1962, standardized
\item FORTRAN 66 — appeared in 1966 as an ANSI standard
\item FORTRAN 77 — appeared in 1977, structured features
\item \textbf{Fortran 90} — 1992 ANSI standard, free form, modules
\item \textbf{Fortran 95} — a few extensions
\item Fortran 2003 — object oriented programming
\item Fortran 2008 — a few extensions
\end{itemize}
The correct spelling of Fortran for 1992 ANSI standard and later
(sometimes called Modern Fortran) is ``Fortran''. Older standards are
spelled as ``FORTRAN''.
}
