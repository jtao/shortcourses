\begin{frame}[fragile]{Simple I/O}
Any program needs to be able to read input and write output to
be useful and portable.
\begin{block}{Simple output with \textbf{PRINT}}
\begin{lstlisting}
print *, <var1> [, <var2> [, ... ]]
\end{lstlisting}
\end{block}
\begin{block}{Simple input with \textbf{READ}}
\begin{lstlisting}
read *, <var1> [, <var2> [, ... ]]
\end{lstlisting}
\end{block}
The * indicates that the data is unformatted.
\end{frame}


\begin{frame}[fragile]{Example with Simple I/O}
\begin{block}{Interactive hello world via I/O}
\begin{lstlisting}
PROGRAM hello
  IMPLICIT NONE
  character(len=100) :: your_name
  print *, 'Your Name Please'
  read *, your_name
  print *, 'Hello ', your_name
END PROGRAM hello
\end{lstlisting}
\end{block}
\end{frame}

\begin{frame}[fragile]{I/O with Unit Number}
Files are identified by some form of file handle, in Fortran called
the unit number. 
\begin{itemize}
 \item The default unit number 5 is associated with the standard input,
 \item Unit number 6 is assigned to standard output.
\end{itemize}
\begin{block}{Read and write through unit number}
\begin{lstlisting}
read(unit,*)
write(unit,*)
\end{lstlisting}
\end{block}
\end{frame}

\begin{frame}[fragile]{File Operations - I}
Fortran provides functions to open, read, write, inquire, and close files.
A file may be opened with the statement
\begin{lstlisting}
OPEN([UNIT=]un, FILE=fname [, options])
READ(un, options)varlist
WRITE(un, options)varlist
INQUIRE([UNIT=]un, options)
CLOSE([UNIT=]un [, options])
\end{lstlisting}
\end{frame}

\begin{frame}[fragile]{File Operations - II}
If data is read/written from/to standard input/output, then
\begin{itemize}
 \item the unit number can also be replaced with *.
 \item use alternate form for reading and writing i.e. the READ *, 
  and PRINT start mentioned earlier.
 \item If data is unformatted i.e. plain ASCII characters, the option to
  WRITE and READ command is *.
\end{itemize}
\end{frame}

\begin{frame}[fragile]{Formatted I/O}
A formatted data description must adhere to the generic form: \textbf{nCw.d}
\begin{itemize}
 \item n is an integer constant that specifies the number of repititions
(default 1 can be omitted),
 \item C is a letter indicating the type of the data variable to be written or
read,
 \item w is the total number of spaces allocated to this variable, and,
 \item d is the number of spaces allocated to the fractional part of the
  variable. Integers are padded with zeros for a total width of w
  provided $d \leq w$.
 \item The decimal (.) and d designator are not used for integers,
characters or logical data types.
\end{itemize}
\end{frame}


\begin{frame}[fragile]{FORMAT Statement - I}
In the simplest form, the format is enclosed in single quotes and
parentheses as argument to the keyword.
\begin{lstlisting}
print '(I5,5F12.6)', i, a, b, c, z ! complex z
write(6, '(2E15.8)') arr1, arr2
read(5, '(2a)') firstname, lastname
\end{lstlisting}
If the same format is to be used repeatedly or it is complicated,
the FORMAT statement can be used. The \textbf{FORMAT} statement must be 
labeled and the label is used in the input/output statement to reference it
\begin{lstlisting}
label FORMAT(formlist)
PRINT label, varlist
WRITE(un, label) varlist
READ(un, label) varlist
\end{lstlisting}
\end{frame}

\begin{frame}[fragile]{FORMAT Statement - II}
The \textbf{FORMAT} statements can occur anywhere in the same program
unit. Most programmers list all \textbf{FORMAT} statements immediately
after the type declarations before any executable statements.
\begin{block}{Format statement examples}
\begin{lstlisting}
10 FORMAT(I5,5F12.6)
20 FORMAT(2E15.8)
100 FORMAT(2a)
print 10, i, a, b, c, z ! complex z
write(6,20) arr1, arr2
read(5,100) firstname, lastname
\end{lstlisting}
\end{block}
\end{frame}