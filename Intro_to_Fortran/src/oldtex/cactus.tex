\frame { \frametitle{Overview}
  \begin{wrapfigure}{R}{0.15\textwidth}
    \hspace{-20pt}
    \begin{center}
      \includegraphics[width=0.15\textwidth]{cactus_logo}
    \end{center}
    \vspace{-20pt}
  \end{wrapfigure}
  $ $
  \textbf{Cactus} is
  \begin{itemize}
   \item a computational framework for developing portable, modular applications solving partial differential equations.
   \item focusing, although not exclusively, on high-performance simulation codes.
   \item designed to allow domain experts in one field to develop
         modules that are transparent to experts in other fields.
  \end{itemize}
}

\frame { \frametitle{Application View}
The structure of an \textbf{application} that is built upon the Cactus computational framework
\begin{figure}
\vspace{0.0cm}
\hspace{0.0cm}
\centering
\includegraphics[width=0.7\textwidth]{cactus_appview}
\end{figure}
}

\frame { \frametitle{Portability \& Scalability}
\footnotesize
\begin{block}{Portability}
  \begin{itemize}
    \item Cactus runs on \textbf{all variants} of the Unix operating system,
          Windows platform, Xbox, etc..
  \end{itemize} 
\end{block}
\begin{block}{Scalability}
  \begin{itemize}
    \item Cactus scales to \textbf{131,072} out of 163,840 cores on
          Intrepid (Blue Gene/P) at ANL with a uniform grid driver.
  \end{itemize}
\end{block}
\normalsize
\begin{figure}
\vspace{0.0cm}
\centering
\includegraphics[width=0.8\textwidth]{bgp_bssn_pugh}
\end{figure}
}