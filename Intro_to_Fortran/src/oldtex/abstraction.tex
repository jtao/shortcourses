\frame { \frametitle{Overview of CaKernel}
\textbf{CaKernel is}
\begin{itemize}
  \item a kernel abstraction;
  \item a parallel programming framework suitable for solving some types of PDEs;
  \item a collection of Cactus modules/thorns;
  \item able to automatically generate and execute CUDA, OpenCL, and C code;
  \item the outcome of our collaborative research efforts with PSNC and other institutes.
\end{itemize}
\textbf{CaKernel is not}
\begin{itemize}
  \item designed to be a generic solution;
  \item in its final form yet.
\end{itemize}
}
\frame { \frametitle{Grid Abstractions behind Cactus \& CaKernel}
\begin{itemize}
\item \textbf{Grid Hierarchy (GH)} represents the distributed adaptive GH.
In Cactus, grid operations are usually handled by a driver thorn to 
create, operate and destroy hierarchical grid structures.
\item \textbf{Grid Function (GF)} represents a distributed data structure that
represents the variables in an application. The application developers are 
responsible for providing proper routines to do initialization, boundary updates, etc.
\item \textbf{Grid Geometry (GG)} represents the coordinates, bounding boxes,
and bounding box lists of the computational domain. Operations on the GG, such
as union, intersection, refine, and coarsen are usually implemented in a driver 
thorn as well.
\end{itemize}
}

\frame { \frametitle{Design of CaKernel}
CaKernel contains $3$ major parts:
\begin{itemize}
\item \textbf{CaKernel Descriptor}: is used to declare the variables that will
be needed in the computation, and identify a few relevant properties;
\item \textbf{CaKernel Templates}: are sets of templates which are highly
optimized for particular types of computational tasks and optimization
strategies;
\item \textbf{CaKernel Code Generator}: is used to parse the descriptors and
automatically generate header files by referring to CaKernel templates. The
descriptor parser and code generator are built on Piraha
(http://code.google.com/p/piraha-peg/).
\end{itemize}
}

\frame { \frametitle{Code Generation}
The \textbf{CaKernel code generator} parses the CaKernel descriptor and automatically generate
CaKernel code from a set of highly optimized templates.
\begin{figure}
\vspace{0.0cm}
\hspace{0.0cm}
\centering
\includegraphics[width=0.8\textwidth]{CaKernelCodeGen}
\end{figure}
}

\frame { \frametitle{3D Stencil Computation}

\begin{figure}
\vspace{0.0cm}
\hspace{0.0cm}
\centering
\includegraphics[width=0.9\textwidth]{algorithm.png}
\end{figure}
\begin{flushright}
 (credit to P. Micikevicius from NVIDIA)
\end{flushright}
}

\frame { \frametitle{Code Workflow}
Cactus variables $\textbf{U}(n)$ are evolved to the next time step $\textbf{U}(n+1)$ during the execution stage.
\begin{figure}
\vspace{0.0cm}
\hspace{0.0cm}
\centering
\includegraphics[width=0.8\textwidth]{CaKernelWorkflow}
\end{figure}
}

\frame { \frametitle{Sample Descriptor}
\begin{figure}
\vspace{0.0cm}
\hspace{0.0cm}
\centering
\includegraphics[width=0.6\textwidth]{CaKernelDescSample}
\end{figure}
}

\frame { \frametitle{Sample Template}
\begin{figure}
\vspace{0.0cm}
\hspace{0.0cm}
\centering
\includegraphics[width=0.6\textwidth]{CaKernelTemplateSample}
\end{figure}
}

\frame { \frametitle{Sample Application Code}
\begin{figure}
\vspace{0.0cm}
\hspace{0.0cm}
\centering
\includegraphics[width=0.6\textwidth]{CaKernelAppSample}
\end{figure}
}