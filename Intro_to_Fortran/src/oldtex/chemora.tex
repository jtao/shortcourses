\frame { \frametitle{Chemora for Heterogeneous Systems}
To carry our success to next generation petascale/exascale heterogeneous systems, 
based on the \textbf{Cactus Computational Framework}, we started the \textbf{Chemora} project. 
\begin{figure}
\vspace{0.0cm}
\hspace{0.0cm}
\centering
\includegraphics[height=0.4\textwidth]{chemora.pdf}
\end{figure}
}

\frame { \frametitle{Cactus Computational Framework}
  \begin{wrapfigure}{R}{0.15\textwidth}
    \begin{center}
      \includegraphics[width=0.15\textwidth]{cactus_logo}
    \end{center}
  \end{wrapfigure}
  $ $
  \textbf{Cactus} is
  \begin{itemize}
   \item a computational framework for developing portable, modular applications solving partial differential equations.
   \item focusing, although not exclusively, on high-performance simulation codes.
   \item designed to allow domain experts in one field to develop
         modules that are transparent to experts in other fields.
   \item supporting adaptive mesh refinement via the Carpet library.
   \item available at \url{http://cactuscode.org/}
  \end{itemize}
}

\frame { \frametitle{Carpet Adaptive Mesh Refinement Library}
  \begin{wrapfigure}{R}{0.15\textwidth}
    \hspace{-20pt}
    \begin{center}
      \includegraphics[width=0.15\textwidth]{carpet_logo}
    \end{center}
    \vspace{-20pt}
  \end{wrapfigure}
  $ $

\textbf{Carpet} is
  \begin{itemize}
    \item a driver layer of Cactus providing adaptive mesh refinement,
          multi-patch capability, as well as parallelization and efficient I/O.
    \item written primarily in C++.
    \item led by Erik Schnetter at LSU and Perimeter Institute.
    \item available at \url{http://carpetcode.org/}
  \end{itemize}
}


\frame { \frametitle{Accelerator Framework}
The naive ``copy to - compute - copy back" approach in GPU programming very often fails to give optimal performance. As a module (or thorn) in Cactus, we developed an accelerator framework.
  \begin{itemize}
    \item We track which data (and what part of the data) are
read and written by a particular routine, and where this routine
executes (host or GPU). 
    \item Data are copied only when necessary, and then only those 
portions that are needed. 
    \item Data are not only accessed for computations; 
inter-process synchronization and I/O also access data, and are typically executed on the host.
  \end{itemize}
}


% 
% \frame { \frametitle{Cactus Application View}
% The structure of an \textbf{application} that is built upon the Cactus computational framework
% \begin{figure}
% \vspace{0.0cm}
% \hspace{0.0cm}
% \centering
% \includegraphics[width=0.7\textwidth]{cactus_appview}
% \end{figure}
% }

% \frame { \frametitle{Portability \& Scalability}
% \footnotesize
% \begin{block}{Portability}
%   \begin{itemize}
%     \item Cactus runs on \textbf{all variants} of the Unix operating system,
%           Windows platform, Xbox, etc..
%   \end{itemize} 
% \end{block}
% \begin{block}{Scalability}
%   \begin{itemize}
%     \item Cactus scales to \textbf{131,072} out of 163,840 cores on
%           Intrepid (Blue Gene/P) at ANL with a uniform grid driver.
%   \end{itemize}
% \end{block}
% \normalsize
% \begin{figure}
% \vspace{0.0cm}
% \centering
% \includegraphics[width=0.8\textwidth]{bgp_bssn_pugh}
% \end{figure}
% }


\frame { \frametitle{CaKernel Programming Abstractions}
\textbf{CaKernel is}
\begin{itemize}
  \item a kernel abstraction;
  \item a parallel programming framework suitable for solving some types of PDEs;
  \item a collection of Cactus modules/thorns;
  \item able to automatically generate and execute CUDA, OpenCL, and C code;
  \item the outcome of our collaborative research efforts with PSNC and other institutes.
\end{itemize}
\textbf{CaKernel is not}
\begin{itemize}
  \item designed to be a generic solution;
  \item in its final form yet.
\end{itemize}
}


\frame { \frametitle{Design of CaKernel}
CaKernel contains $3$ major parts:
\begin{itemize}
\item \textbf{CaKernel Descriptor}: is used to declare the variables that will
be needed in the computation, and identify a few relevant properties;
\item \textbf{CaKernel Templates}: are sets of templates which are highly
optimized for particular types of computational tasks and optimization
strategies;
\item \textbf{CaKernel Code Generator}: is used to parse the descriptors and
automatically generate header files by referring to CaKernel templates. The
descriptor parser and code generator are built on Piraha
(http://code.google.com/p/piraha-peg/).
\end{itemize}
}


\frame { \frametitle{Grid Abstractions behind Cactus \& CaKernel}
\begin{itemize}
\item \textbf{Grid Hierarchy (GH)} represents the distributed adaptive GH.
In Cactus, grid operations are usually handled by a driver thorn to 
create, operate and destroy hierarchical grid structures.
\item \textbf{Grid Function (GF)} represents a distributed data structure that
represents the variables in an application. The application developers are 
responsible for providing proper routines to do initialization, boundary updates, etc.
\item \textbf{Grid Geometry (GG)} represents the coordinates, bounding boxes,
and bounding box lists of the computational domain. Operations on the GG, such
as union, intersection, refine, and coarsen are usually implemented in a driver 
thorn as well.
\end{itemize}
}


\frame { \frametitle{CaKernel Code Generation}
The \textbf{CaKernel code generator} parses the CaKernel descriptor and automatically generate
CaKernel code from a set of highly optimized templates.
\begin{figure}
\vspace{0.0cm}
\hspace{0.0cm}
\centering
\includegraphics[width=0.7\textwidth]{CaKernelCodeGen}
\end{figure}
}

\frame { \frametitle{3D Stencil Computation}
\begin{figure}
\vspace{0.0cm}
\hspace{0.0cm}
\centering
\includegraphics[width=0.7\textwidth]{algorithm.png}
\end{figure}
\begin{flushright}
 \textsc{(credit to P. Micikevicius from NVIDIA)}
\end{flushright}
}

\frame { \frametitle{Code Workflow}
Cactus variables $\textbf{U}(n)$ are evolved to the next time step $\textbf{U}(n+1)$ during the execution stage.
\begin{figure}
\vspace{0.0cm}
\hspace{0.0cm}
\centering
\includegraphics[width=0.6\textwidth]{CaKernelWorkflow}
\end{figure}
}


\frame { \frametitle{Kranc Code Generation Package}
  \begin{wrapfigure}{R}{0.15\textwidth}
    \hspace{-20pt}
    \begin{center}
      \includegraphics[width=0.15\textwidth]{kranc_logo}
    \end{center}
    \vspace{-20pt}
  \end{wrapfigure}
  $ $

\textbf{Kranc} is
\begin{itemize}
  \item a suite of Mathematica packages with a computer algebra
        toolbox for numerical relativists.
  \item a prototyping system for physicists or mathematicians
        handling very complicated systems of partial differential equations.
  Kranc can generate entire Cactus based codes starting from a high
  level set of partial differential equations.
  \item available at \url{http://kranccode.org/}
\end{itemize}
}

\frame { \frametitle{Integration with Cactus}
Kranc is closely coupled with Cactus by
\begin{itemize}
\item Defining the grid functions which the simulation will use.
\item Performing a user-specified calculation at each point of the grid.
\item Computing the right hand sides of evolution equations so that the
time integrator can compute the evolved variables at the next time step.
\end{itemize}
}

\frame { \frametitle{Potential Applications}
Various kinds of initial value boundary problems:
\begin{block}{}
\begin{equation}
\begin{array}{lcl}
  \partial_t u^a &=& f(u^a), \\
  u^a |_{t=0} &=& g(u^a), \\
  u^a |_{\partial \Sigma} &=& h(u^a) \nonumber
\end{array}
\end{equation}
\end{block}
  \begin{figure}
    \begin{center}
      \includegraphics[width=0.4\textwidth]{boundary}
    \end{center}
    \vspace{-20pt}
  \end{figure}
}

\frame{
\frametitle{Kranc in Action: Wave Equation (I)}
\begin{block}{Mathematical expression:}
\begin{equation}
\partial_t^2 u = \delta^{ab}\partial_b\partial_a u  \nonumber
\end{equation}
\end{block}
\begin{block}{Rewritten in 1st order in time:}
\begin{equation}
\begin{array}{lcl}
\partial_t u & = & \rho, \\
\partial_t \rho & = & \delta^{ab}\partial_b\partial_a u \nonumber
\end{array}
\end{equation}
\end{block}
\begin{block}{Input to Kranc:}
\begin{equation}
\begin{array}{lcl}
    dot[u] &\rightarrow & rho, \\
    dot[rho] &\rightarrow & KD[ua,ub] PD[u[la],lb] \nonumber
\end{array}
\end{equation}
\end{block}
}

\begin{frame}[fragile]
\frametitle{Kranc in Action: Wave Equation (II)}
\begin{block}{Generated C code to calculate the RHS:}
\begin{verbatim}
/* Precompute derivatives (new style) */
    PDstandardNth11u = PDstandardNth11(u, i, j, k);
    PDstandardNth22u = PDstandardNth22(u, i, j, k);
    PDstandardNth33u = PDstandardNth33(u, i, j, k);
/* Calculate temporaries and grid functions */
    urhsL = rhoL;
    rhorhsL = PDstandardNth11u + PDstandardNth22u
            + PDstandardNth33u;
/* Copy local copies back to grid functions */
    rhorhs[index] = rhorhsL;
    urhs[index] = urhsL;
\end{verbatim}
\end{block}
\end{frame}

\frame { \frametitle{McLachlan Code}
The McLachlan code is
\begin{itemize}
  \item a public code created in the XiRel project to solve the EFE in vacuum,
        namely
        \\$$G_{\mu\nu} \equiv R_{\mu\nu} - \frac{1}{2}g_{\mu\nu} R = 0.$$
  \item used routinely for binary blackhole simulations.
  \item automatically generated using Kranc.
  \item coupled with LoopControl for hybrid MPI + OpenMP parallel programming.
\end{itemize}
}

\frame { \frametitle{$\tilde R_{ij}$ in McLachlan (I)}
In the BSSN (Baumgarte-Shapiro-Shibata-Nakamura) $3+1$ formalism used in
McLachlan, we need to calculate the
3D conformal part of Ricci tensor $\tilde R_{ij}$.
\begin{block}{Mathematical expression:}
\begin{eqnarray}
\tilde R_{ij} & = & - \frac{1}{2} \tilde \gamma^{lm}
        \tilde \gamma_{ij,lm}
        + \tilde \gamma_{k(i} \partial_{j)} \tilde \Gamma^k
        + \tilde \Gamma^k \tilde \Gamma_{(ij)k} \nonumber \\
       &+& \tilde \gamma^{lm} \left( 2 \tilde \Gamma^k_{l(i}
        \tilde \Gamma_{j)km} + \tilde \Gamma^k_{im} \tilde \Gamma_{klj}
        \right) \nonumber
\end{eqnarray}
\end{block}
}

\frame { \frametitle{$\tilde R_{ij}$ in McLachlan (II)}
\begin{block}{Input to Kranc:}
\begin{equation}
\begin{array}{lcl}
    Rt[li,lj] &\rightarrow &  - (1/2) gtu[ul,um] PD[gt[li,lj],ll,lm] \\
                & + & (1/2) gt[lk,li] PD[Xt[uk],lj] \\
                & + & (1/2) gt[lk,lj] PD[Xt[uk],li] \\
                & + & (1/2) Xtn[uk] gt[li,ln] Gt[un,lj,lk] \\
                & + & (1/2) Xtn[uk] gt[lj,ln] Gt[un,li,lk] \\
                & + & gtu[ul,um] (+ Gt[uk,ll,li] gt[lj,ln] Gt[un,lk,lm] \\
                & + & Gt[uk,ll,lj] gt[li,ln] Gt[un,lk,lm] \\
                & + & Gt[uk,li,lm] gt[lk,ln] Gt[un,ll,lj]) \nonumber
\end{array}
\end{equation}
\end{block}
}

\frame { \frametitle{$\tilde R_{ij}$ in McLachlan (III)}
Generated C code to calculate $\tilde R_{11}$ (5 more not shown)
  \begin{figure}
    \begin{center}
      \includegraphics[width=0.7\textwidth]{Rt11}
    \end{center}
    \vspace{-20pt}
  \end{figure}
}

\frame { \frametitle{McLachlan in Action}
McLachlan is also
\begin{itemize}
  \item the numerical kernel of the new Cactus benchmark submitted to the SPEC CPU Benchmark Search Program.
  \item adopted by the LSU team to compete in the Second IEEE International
        Scalable Computing Challenge (SCALE 2009) at CCGrid 2009.
\end{itemize}
}

\frame { \frametitle{Kranc - CaKernel GPU Code Optimization}
The CaKernel parts of Chemora use Kranc-provided
and run-time-available information to generate efficient GPU
executables from the numerical kernel.
\begin{itemize}
\item Stencils and dynamic tile selection (loop tiling, heuristic approaches)
\item Lightweight kernel generation (dynamic code generation, index arithmetic simplification)
\item Fat kernel detection (loop fusion, code reconstruction, dynamic adjustment of the number of threads)
\item Integrated performance monitoring (PAPI, NVIDIA Cupti, kernel identification)
\end{itemize}
}


\frame { \frametitle{Equation Description Language}
\begin{itemize}
  \item Very high level, Latex-like syntax for domain scientists
  \item Description: variables, equations, initial/boundary conditions, parameters, analysis quatities
  \item A layer between real physics problems to numerical implementations
\end{itemize}
\begin{figure}
\vspace{0.0cm}
\hspace{0.0cm}
\centering
      \includegraphics[width=0.4\textwidth]{edl.png}
\end{figure}
}

\frame { \frametitle{Binary Blackhole Simulation}
Modeling gravitational waves from binary blackhole system.
\begin{figure}
\vspace{0.0cm}
\hspace{0.0cm}
\centering
\includegraphics[width=0.4\textwidth]{bbhsim.png}
\end{figure}
}
