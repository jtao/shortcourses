\frame { \frametitle{Adaptive Mesh Refinement Method}
  The B\&O AMR algorithm proposed by Berger and Oliger in 1984 is
  built upon a nested grid hierarchy of rectangular grids with increasing resolution.
  \begin{figure}
    \vspace{0.0cm}
    \hspace{0.0cm}
    \centering
    \includegraphics[width=0.8\textwidth]{gridhierarchy}
  \end{figure}
}


\frame { \frametitle{Domain Decomposition in AMR}
 \begin{centering}\includegraphics[width=11cm]{domain_decomposition_mr}\\\end{centering}
}

\frame { \frametitle{Carpet Adaptive Mesh Refinement Library}
  \begin{wrapfigure}{R}{0.2\textwidth}
    \begin{center}
      \includegraphics[width=0.2\textwidth]{carpet_logo}
    \end{center}
    \vspace{-20pt}
  \end{wrapfigure}
  $ $

Carpet
  \begin{itemize}
    \item is a driver layer of Cactus providing adaptive mesh refinement,
          multi-patch capability, as well as parallelization and efficient I/O.
    \item is written primarily in C++.
    \item was created in 2001 by Erik Schnetter at the Theoretische Astrophysik T\"{u}bingen.
    \item is currently maintained at the Center for Computation \& Technology at LSU.
  \end{itemize}
}

\frame { \frametitle{Scalability of Carpet}
Carpet has been actively developed, and it is now
used by several numerical relativity groups around the world for their production
simulations of binary compact objects.
  \begin{figure}
    \begin{center}
      \includegraphics[width=0.48\textwidth]{carpet_scale}
      \includegraphics[width=0.48\textwidth]{carpet_scale_bgp}
    \end{center}
  \end{figure}
}