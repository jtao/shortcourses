\begin{frame}[fragile]{Control Constructs}
A Fortran program is executed sequentially. Control Constructs change the sequential execution order of the
program.
\begin{itemize}
 \item Conditionals: \textbf{IF}, \textbf{IF-THEN-ELSE}
 \item Switches: \textbf{SELECT/CASE} 
 \item Loops: \textbf{DO}
 \item Branches: \textbf{GOTO} (obsolete in Fortran 95/2003, use CASE instead)
\end{itemize}
\end{frame}

\begin{frame}[fragile]{Conditionals}
\begin{block}{IF construct}
\begin{lstlisting}
if ( expression ) statement
\end{lstlisting}
\end{block}
\begin{block}{IF THEN ELSE construct}
\begin{lstlisting}
if ( expression 1 ) then
  executable statements
else if ( expression 2 ) then
  executable statements
else
  executable statements
end if
\end{lstlisting}
\end{block}
\end{frame}

\begin{frame}[fragile]{Conditionals - IF Example}
\begin{block}{IF construct}
\begin{lstlisting}
if (value < 0) value = 0
\end{lstlisting}
\end{block}
\begin{itemize}
 \item When the if statement is executed, the logical expression is evaluated.
 \item If the result is true, the statement following the logical
expression is executed; otherwise, it is not executed.
 \item The statement following the logical expression cannot be
another if statement. Use the if-then-else construct instead.
\end{itemize}

\end{frame}

\begin{frame}[fragile]{Conditionals - IF-THEN-ELSE Example}
\begin{block}{IF THEN ELSE construct}
\begin{lstlisting}
if ( x < 50 ) then
  GRADE = 'F'
else if ( x >= 50 .and. x < 60 ) then
  GRADE = 'D'
else if ( x >= 60 .and. x < 70 ) then
  GRADE = 'C'
else if ( x >= 70 .and. x < 80 ) then
  GRADE = 'B'
else
  GRADE = 'A'
end if
\end{lstlisting}
\end{block}
\end{frame}

\begin{frame}[fragile]{Switches}
\begin{block}{SELECT CASE construct}
\begin{lstlisting}
[case_name:] select case ( expression )
  case ( selector )
    executable statement
  case ( selector )
    executable statement
  case default
    executable statement
end select [case_name]
\end{lstlisting}
\end{block}
The value of the expression in the select case should be an integer or a character string.
The case name is optional.
\end{frame}

\begin{frame}[fragile]{Switches - Example I}
\begin{block}{Character case selector}
\begin{lstlisting}
select case ( traffic_light )
  case ( "red" )
    print *, "Stop"
  case ( "yellow" )
    print *, "Caution"
  case ( "green" )
    print *, "Go"
  case default
    print *, "Illegal value: ", traffic_light
end select
\end{lstlisting}
\end{block}
\end{frame}

\begin{frame}[fragile]{Switches - Example II}
\begin{block}{Integer case selector}
\begin{lstlisting}
select case ( score )
  case ( 50 : 59 )
    GRADE = "D"
  case ( 60 : 69 )
    GRADE = "C"
  case ( 70 : 79 )
    GRADE = "B"
  case ( 80 : )
    GRADE = "A"
  case default
    GRADE = "F"
end select
\end{lstlisting}
\end{block}
\end{frame}


\begin{frame}[fragile]{Loops - DO}
\begin{block}{DO construct}
\begin{lstlisting}
[do name:] do loop_control
  execution statements
end do [do name]
\end{lstlisting}
\end{block}
The do loop name is optional. To exit the do loop, use the \textbf{EXIT} 
or \textbf{CYCLE} statement.
\begin{itemize}
 \item The \textbf{EXIT} statement causes termination of execution of a loop.
 \item The \textbf{CYCLE} statement causes termination of the execution of one
iteration of a loop.
\end{itemize}
\end{frame}

\begin{frame}[fragile]{Loops - DO Example}
\begin{block}{Factorial with DO construct}
\begin{lstlisting}
program factorial1
  implicit none
  integer(KIND=8) :: i,factorial, n=6
  factorial = n
  do i = n-1,1,-1
    factorial = factorial * i
  end do
  write(*,'(i4,a,i15)') n,'!=',factorial
end program factorial1
\end{lstlisting}
\end{block}
\end{frame}

\begin{frame}[fragile]{Loops - DO WHILE}
If a condition is to be tested at the top of a loop, a do ... while
loop can be used
\begin{block}{DO WHILE construct}
\begin{lstlisting}
[do name:] do while ( expression )
  executable statements
end do [do name]
\end{lstlisting}
\end{block}
The loop only executes if the logical expression is \textbf{.TRUE.}
\end{frame}

\begin{frame}[fragile]{Loops - DO WHILE Example}
\begin{block}{DO WHILE example}
\begin{lstlisting}
finite: do while ( i <= 100 )
  i = i + 1
  inner: if ( i < 10 ) then
    print *, i
  end if inner
end do finite
\end{lstlisting}
\end{block}
\end{frame}