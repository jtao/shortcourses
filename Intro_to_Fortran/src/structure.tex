\begin{frame}[fragile]{Program - I}
\begin{block}{Typical Fortran program structure}
\begin{lstlisting}
PROGRAM program-name
  IMPLICIT NONE
  [specification part]
  [execution part]
  [subprogram part]
END PROGRAM program-name
\end{lstlisting}
\end{block}
The \textbf{PROGRAM} statement (optional) gives a name to the program. 
The first character of the name must be a letter. Use the \textbf{IMPLICIT NONE} statement to
avoid implicit typing rules. The \textbf{END} statement terminates the program and returns control to the 
computer's operating system.
\end{frame}

\begin{frame}[fragile]{Program - II}
A Fortran program consists of one or more program units:
\textbf{PROGRAM, SUBROUTINE, FUNCTION, MODULE}
\begin{itemize}
 \item The unit containing the \textbf{PROGRAM} attribute is often called the main
program or main.
 \item The main program should begin with the \textbf{PROGRAM} keyword (not required, but recommended).
 \item A Fortran program should contain only one main program and one or more subprogram
units such as \textbf{SUBROUTINE}, \textbf{FUNCTION} and \textbf{MODULE}.
 \item Every program unit, must end with a \textbf{END} keyword.
\end{itemize}
\end{frame}


\begin{frame}[fragile]{An Example}
\begin{block}{Temperature Conversion between Fahrenheit and Celsius}
\begin{lstlisting}
program temp
  implicit none
  real :: tempC, tempF
! Convert 10C to fahrenheit
  tempF = 9.0 / 5.0 * 10.0 + 32.0
! Convert 40F to celsius
  tempC = 5.0 / 9.0 * (40.0 - 32.0 )
  call display(tempc, tempF)
end program temp
\end{lstlisting}
\end{block}
\end{frame}


\begin{frame}[fragile]{Subroutines - I}
\begin{block}{Typical Fortran subroutine structure}
\begin{lstlisting}
SUBROUTINE subroutine-name(dummy arguments)
  IMPLICIT NONE
  [specification part]
  [execution part]
  [subprogram part]
END SUBROUTINE subroutine-name
\end{lstlisting}
\end{block}
\end{frame}

\begin{frame}[fragile]{Subroutines - I}
\begin{block}{Typical Fortran subroutine structure}
\begin{lstlisting}
SUBROUTINE subroutine-name(dummy arguments)
  IMPLICIT NONE
  [specification part]
  [execution part]
  [subprogram part]
END SUBROUTINE subroutine-name
\end{lstlisting}
\end{block}
\end{frame}

\begin{frame}[fragile]{Subroutines - II}
\begin{itemize}
 \item CALL Statement:
 \begin{itemize}
  \item The  \textbf{CALL} statement evaluates its arguments and transfers control to the subroutine
  \item Upon return, the next statement is executed.  
 \end{itemize}
 \item SUBROUTINE Statement:
 \begin{itemize}
  \item The \textbf{SUBROUTINE} statement declares the procedure and its arguments.
  \item These are also known as dummy arguments.
 \end{itemize}
 \item The subroutine’s interface is defined by
 \begin{itemize}
  \item The subroutine statement itself
  \item The declarations of its dummy arguments
  \item Anything else that the subroutine uses
 \end{itemize}
\end{itemize}
\end{frame}


\begin{frame}[fragile]{A Subroutine Example}
\begin{block}{Calcuate the sum of input variables}
\begin{lstlisting}
SUBROUTINE calc(a,b,c, sum)
  IMPLICIT NONE
  real :: a,b,c,sum
  sum = a + b + c
  return
END SUBROUTINE calc
\end{lstlisting}
\end{block}
\end{frame}

\begin{frame}[fragile]{Functions}
Fortran functions operate on the same principle as subroutines.
The only difference is that function returns a value and does not
involve the call statement.

\begin{block}{Calculate the sum of the input variables}
\begin{lstlisting}
FUNCTION calc(a,b,c)
  IMPLICIT NONE
  real :: a,b,c,calc
  calc = a + b + c
END FUNCTION calc
\end{lstlisting}
\end{block}
\end{frame}


\begin{frame}[fragile]{Recursive Functions}
In Fortran 90, recursive procedures call themselves and must be declared explicitly.
The recursive function declarations must contain a result keyword, and 
one type of declaration refers to both the function name and the result variable.
\begin{block}{Calcuate the factorial}
\begin{lstlisting}
RECURSIVE FUNCTION fact(i) result(i_fact)
integer :: i, i_fact
if ( i > 0 ) then
  i_fact = i * fact(i - 1)
else
  i_fact = 1
end if
END FUNCTION fact
\end{lstlisting}
\end{block}
\end{frame}

\begin{frame}[fragile]{Modules}
A module is a program unit whose functionality can be exploited
by other programs which attaches to it via the \textbf{USE} statement.
\begin{block}{Example of module}
\begin{lstlisting}
MODULE PRECISION
 IMPLICIT NONE
 integer, parameter :: dp = 8
END MODULE PRECISION
\end{lstlisting}
\end{block}
\end{frame}
